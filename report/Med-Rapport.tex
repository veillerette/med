\documentclass[12pt]{report}

\usepackage[utf8]{inputenc}
\usepackage[T1]{fontenc}
\usepackage{graphicx}
\usepackage[french]{babel} 
\usepackage[top=3cm, bottom=3cm, left=3cm, right=3cm]{geometry}
\usepackage{tikz}
\usepackage{color}
\usepackage{colortbl}
\usepackage[colorlinks=true,linktocpage]{hyperref}
\usepackage{hyperref}
\usepackage{listings}
\usepackage{amsmath}
\usepackage{diagbox}
\usepackage{framed}
\usepackage{float}
\usepackage{amsmath}
\usepackage{amssymb}
\usepackage{mdframed}

\definecolor{mygray}{rgb}{0.5,0.5,0.5}
\lstdefinestyle{CLangage}{
language=C,
basicstyle=\normalsize,
backgroundcolor=\color[rgb]{0.95,0.95,0.95},
upquote=true,
xleftmargin=30pt,
framexleftmargin=13pt,
        framextopmargin=3pt,
        framexbottommargin=3pt, 
        frame=tb, framerule=0pt,
aboveskip={0.2\baselineskip},
showstringspaces=true,
extendedchars=true,
breaklines=true,
showtabs=false,
showspaces=false,
showstringspaces=false,
identifierstyle=\ttfamily,
keywordstyle=\color[rgb]{0,0,1},
commentstyle=\footnotesize\color[rgb]{0.133,0.545,0.133},
stringstyle=\color[rgb]{0.627,0.126,0.941},
numbers=left,              
  numbersep=5pt,
   numberstyle=\tiny\color{mygray},
   stepnumber=1,
   morekeywords={X:,T:,C:,M:,L:,Q:,K:,V:},
}
\lstdefinestyle{Bash}{
language=Bash,
basicstyle=\normalsize,
backgroundcolor=\color[rgb]{0.95,0.95,0.95},
upquote=true,
xleftmargin=30pt,
framexleftmargin=13pt,
        framextopmargin=3pt,
        framexbottommargin=3pt, 
        frame=tb, framerule=0pt,
aboveskip={0.2\baselineskip},
columns=fullflexible,
showstringspaces=true,
extendedchars=true,
breaklines=true,
showtabs=false,
showspaces=false,
showstringspaces=false,
identifierstyle=\ttfamily,
keywordstyle=\color[rgb]{0,0,1},
commentstyle=\footnotesize\color[rgb]{0.133,0.545,0.133},
stringstyle=\color[rgb]{0.627,0.126,0.941},
numbers=none,                    
  numbersep=5pt,
   numberstyle=\tiny\color{mygray},
   stepnumber=1
}
\definecolor{grey}{rgb}{0.95, 0.95, 0.95}


\begin{document}



\title{\begin{Huge}Editeur musical avec lecture audio\end{Huge} \\
    \begin{normalsize}\textit{et génération analogique d'ondes sinusoïdales et carrées}\end{normalsize}
    }
\author{Victor \bsc{Veillerette}}
\date{Février 2017}
 
\maketitle
\addtocontents{toc}{\vspace*{-2\baselineskip}}
\tableofcontents

\chapter{Introduction}
\section{Sujet initial}
Le sujet de base était la réalisation d'un éditeur de partitions générant ses propres sons.\\
Le projet était à réaliser durant un semestre, c'est-à-dire d'Octobre 2016 à Février 2017. \\
Voici l'intitulé exact du sujet :\\
\begin{mdframed}[backgroundcolor=grey]
 


\begin{quotation}
Le but du projet est de réaliser un éditeur classique de partition
musicale similaire à MuseScore, proposant les
fonctionnalités basiques d'édition d'une partition musicale.\\
On permettra à l'utilisateur d'éditer une partition musicale en le
forçant à respecter les contraintes inhérentes (entre autre: l'empêcher
de mettre trop de notes par rapport à la durée d'une mesure), en se
focalisant dans un premier temps sur une seule voix. L'outil devra être
capable de jouer la partition ainsi rédigée (sons générés par ordinateur
- ondes sinusoidales par exemple).\\
L'utilisateur devra en outre pouvoir "personnaliser" les fréquences de
chaque note. Tout en gardant 12 divisions par octave, on pourrait par
exemple, déterminer chaque "quantité de fréquences" à allouer à chaque
division.
\end{quotation}


\end{mdframed}
\vskip 0.3in

Les modalités du projet étaient les suivantes :\\
\begin{itemize}
 \item Langage de programmation : C   \\
 \item Utilisation de la librairie SDL \\
 \item Aucune utilisation de code tiers \\
\end{itemize}


%\section{Analyse des fonctionnalités}

\chapter{Développement} % Sources, Organisation dossiers, Bibliothèques, Histogramme ?
\section{Sources}
Le projet est une application entièrement écrite en C ANSI (1989). 
Il est à noter que le programme n'utilise aucune image externe et génère de lui-même tous les dessins affichés à l'écran. Pour cela le programme utilise les librairies suivantes :\\
\begin{itemize}
 \item \textbf{SDL-1.2} : La librairie \textit{Simple Direct Media Library} permet l'ouverture d'une fenêtre, le remplissage de rectangles, la gestion des événements.\\
 \item \textbf{SDL\_gfx 2.0.25} : Cette extension de la SDL permet l'utilisation de primitives de dessin et de zoom plus évoluées.\\
 \item \textbf{SDL\_ttf 2.0} : Cette extension de la SDL permet l'affichage de texte à l'écran. \\
\end{itemize}
 Le dossier principal du projet est organisé de la façon suivante :\\
\begin{itemize}
 \item \textbf{doxygen/} : Documentation technique générée par Doxygen. Elle regroupe la documentation complète de toutes les structures et de toutes les fonctions du projet.\\
 \item \textbf{media/} : Fichiers de polices nécessaires à l'exécution de l'application.\\
 \item \textbf{include/} : Fichiers headers du projet .h; chaque fichier est rattaché à un fichier source, et donc à un module.\\
 \item \textbf{doc/} : Quelques fichiers exemples .abc ou .med à charger dans l'application.\\
 \item \textbf{src/} : Fichiers sources du projet .c; chaque fichier correspond à un module. \\
 \item \textbf{Makefile} : Fichier Makefile permettant la compilation avec la commande \textit{make}.\\
\end{itemize}
\section{Plateforme de suivi Git}
Malgré le fait que le développement du programme ne soit effectué que par une personne, j'ai voulu tester l'utilisation du logiciel de gestion de version \textbf{git}
avec la plateforme en ligne \textbf{Github.com}\footnote{\url{https://github.com/}}.\\ \\
Le projet est constitué d'environ 130 commits pour plus de 13000 lignes et 30 fichiers :\\

\begin{figure}[H]
  \centering
  \includegraphics[scale=0.58]{git2.png}\\
    \caption{\textit{Evolution du nombre de commits par semaine}}
\end{figure}


\chapter{Implémentations}
\section{Partition}
\subsection{Durée des notes}
La durée des notes est représentée dans le programme comme dans le domaine de la notation musicale, c'est-à-dire qu'une Ronde se note 1, une Blanche 2, une Noire 4, etc.\\
Sachant que le programme représente les durées de base de la Ronde jusqu'à la Quadruple-Croche, il est alors possible de récupérer la durée réelle d'une note grâce à la formule :\\
\[ Real\_Duration(duration) = \frac{64}{duration} \]\\
\lstset{style=CLangage}
\begin{lstlisting}[language=C]
typedef enum {
	RONDE = 1,		BLANCHE	= 2,
	NOIRE = 4,		CROCHE	= 8,
	DOUBLECROCHE = 16, 	TRIPLECROCHE = 32
	QUADRUPLECROCHE = 64
	} Note_Duration;
\end{lstlisting}
\newpage
\subsection{Hauteur des notes}
La hauteur des notes suit la spécification MIDI mais commence à l'octave 0 jusqu'à la fin de l'octave 8 : \\
\begin{center}
\includegraphics[scale=0.50]{midi.jpg}
\end{center}
\vskip 0.3in
\subsection{Structure d'une note}
Une note est représentée par sa hauteur, sa durée, des drapeaux (Altérations, note pointée, liaison). Ici un silence est représenté de la même manière
(certains champs sont alors ignorés). La structure suivante d'une note combinant ces deux aspects est donc utilisée :\\
\lstset{style=CLangage}
\begin{lstlisting}[language=C]
struct Note
{
	char note; 	/* Hauteur */
	char rest; 	/* Note / Silence */
	
	Note_Flags flags : 24;	/* Drapeaux */
	Note_Duration duration : 8;	/* Duree */
};
\end{lstlisting}
\subsection{Représentation d'une mesure}
Une mesure est représentée par un type de mesure (4/4, 3/4 , ...), une clé, des drapeaux, une armure et une suite de note structurée
comme une liste chaînée simple.\\Au départ, une mesure est initialisée avec uniquement des silences (les plus grands possibles).\\ La durée totale
de la mesure (num * den) est conservée à l'ajout et à la suppression de note (qui équivaut à changer le champ rest d'une note).\\ \newpage
Voici la structure pour représenter une mesure :\\
\lstset{style=CLangage}
\begin{lstlisting}[language=C]
struct Step
{
	ToNote *notes;		/* Liste chainees */
	int num;		/* Nombre de temps */
	Cle cle : 16;		/* Cle */
	Note_Duration den : 8;	/* Duree de base */
	Step_Flags flags : 16;	/* Drapeaux */
	signed char sign;	/* Armure */
};
\end{lstlisting}
\vskip 0.3in
L'armure est constituée comme ceci :\\
\begin{itemize}
 \item Si \textbf{sign} vaut 0, alors la tonalité est de DoM/Lam : l'armure est vide.\\
 \item Si \textbf{sign} est négatif, alors l'armure est constituée de |sign| bémols.\\
 \item Si \textbf{sign} est positif, alors l'armure est constituée de sign dièses.\\
\end{itemize}

\subsection{Portée et Partition}
Une portée n'est qu'un tableau de plusieurs mesures les unes à la suite des autres. En effet, la suppression/déplacement d'une mesure est une
opération coûteuse en performance mais rare pour cette utilisation.\\
De la même manière, une partition est un tableau de plusieurs portées. Il est à noter qu'il est évident que chaque portée doit avoir le même nombre
de mesures : Ceci est vérifié par les fonctions d'ajouts/suppressions de mesures qui permettent de garder cet état vrai.\\

\newpage
\section{Audio}
\subsection{Principe}
Il est à noter qu'ici, en aucun cas le programme n'utilise la librairie SDL\_mixer qui faciliterait l'utilisation de
sons MP3/WAV externes mais qui empêcherait la configuration des fréquences jouées.\\
Le programme travaille avec des signaux de caractéristiques suivantes :\\
\begin{itemize}
 \item \textbf{Un taux d'échantillonage} de 44100Hz (qualité CD) : Une seconde d'un signal audio est divisée en 44100 valeurs.\\
 \item des valeurs en \textbf{16 bits signés} \\
 \item \textbf{2 Channels} (Gauche / Droite)\\
 \item Un buffer audio \textbf{de 256 valeurs}. \\
\end{itemize}

\subsection{Fréquences de base}
Pour calculer la fréquence de base d'une note dans la gamme tempérée, le programme effectue le calcul suivant :\\
\begin{large}
\[ 
  f(h) = 33 \times {2}^{(\frac{h}{12} - 2)} \times {1.059463}^{(h \bmod 12)}
\]
\end{large}

\subsection{Remplissage du buffer et Mixage}
La fonction \textbf{main\_callback()} du fichier Audio.c se charge de remplir le buffer une fois que celui-ci est vide.\\ \\
Dans un premier temps, cette fonction va faire appel à une fonction de mixage.\\ Celle-ci va considérer toutes les notes
jouées en cours et additionner leur signal et normaliser le signal résultant.\\ \\ Ce signal de 256 valeurs va ensuite être
copié dans le buffer pour être joué.\\\\

Voici comment calculer un signal à partir d'une fréquence :\\

\begin{figure}[H]
\centering
\[
\begin{aligned}
sinusoide(x, frequence) &  =  \sin(\pi\times \mathit{frequence}\times \frac{x}{44100}) \\
carre(x, frequence)  & =   \lfloor sinusoide(x, \mathit{frequence}) \rfloor   \\
fusion(x, frequence) & =  \mathit{carre}(x, \mathit{frequence}) + \mathit{sinusoide}(x, \mathit{frequence})  \\
harmo(x, frequence) & =  \sum_{i=0}^{4} \mathit{carre}(x, \mathit{frequence}\times 2^i)
\end{aligned}
 \]
\caption{\textit{Génération des signaux}}
\end{figure}

Pour le mixage de plusieurs ondes audios simultanées (plusieurs portées jouant en même temps), le programme effectue la moyenne de toutes les ondes non nulles.

\subsection{Lecture}
Un thread est créé pour chaque portée. Ainsi, chaque thread va analyser la mesure suivante, trouver la note à jouer et l'envoyer au mixeur du processus parent.\\
Les threads utilisés sont les SDL\_Thread gérés automatiquement par la librairie SDL.\\ \\
Pour jouer une note, le programme suit ces étapes :\\
\begin{enumerate}
 \item Récupération de la fréquence tempérée de la note \\
 \item Modification au besoin de la fréquence en fonction de la configuration audio\\
 \item Lancement de la lecture pendant T-10 millisecondes, où T est la durée totale de la note\\
 \item Pause de la lecture pendant 10 millisecondes\\
 \item Passage à la note suivante\\
\end{enumerate}

La pause de la note pendant un temps très court (environ 10ms) est utilisée pour le cas où deux notes de hauteurs identiques seraient mises à la suite et ce pour éviter toute sensation de note liée alors que
ce n'est pas le cas.\\



\newpage
\chapter{Eléments Techniques}
\section{Compilation et Lancement}
Pour compiler le projet, un Makefile est fourni. Il créera le programme \textbf{med}. Pour lancer la compilation, 
il suffit de lancer la commande \textbf{make} et pour retourner à un dossier propre, d'utiliser la commande \textbf{make clean},
ce qui supprimera l'exécutable et les fichiers temporaires liés à la compilation :\\
\lstset{style=Bash}
\begin{lstlisting}
make 		   #Compilation
make clean 	   #Nettoyage
\end{lstlisting}
\vskip 0.1in
Pour lancer l'application il suffit d'exécuter le programme qui vient d'être compilé. Il est aussi possible d'importer un fichier MED ou ABC directement en ligne de commande :\\
\begin{lstlisting}
./med #Lancement simple
./med file.med     #Lancement avec ouverture du fichier MED file.med
./med -abc file.abc  #Lancement avec importation du fichier ABC file.abc
\end{lstlisting}

\section{Format MED}
Le format MED est le format propre du projet. C'est un format binaire qui se base sur la recopie octet par octet de la mémoire dans un fichier. Cela permet une sauvegarde et une ouverture rapide.\\
Ce format est un fichier non compressé, il est donc relativement lourd (quelques ko à 5-6Mo en fonction de la taille de la partition) puisque la plupart des octets du fichier sont à 0 et que le programme
autorise la redondance de symboles. \\

\section{Format ABC}
Le programme permet d'importer ou d'exporter des fichiers au format ABC. Étant donné que c'est un format très libre et très permissif, 
seul le format ABC 2.0\footnote{Documentation ABC 2.0 : \url{http://abcnotation.com/wiki/abc:standard:v2.0}} peut être importé. Mais vu qu'il n'y a pas moyen de différencier les versions,
le programme tentera quand même d'ouvrir tout type de fichier ABC.\\ \\
Toutefois, toutes les fonctionnalités ne sont pas supportées et le programme commence par modifier la syntaxe du fichier en entrée (simplification et création d'un nouveau fichier temporaire),
puis il tente de le parser.\\\\
Voici le format idéal reconnu :\\\vskip 0.3in
Le format se compose d'abord d'un header: \\
\begin{lstlisting}[language=Tex]
X:7959		     % Identifiant unique de melodie (obligatoire)
T:Titre		     % Titre
C:Compositeur	     % Compositeur
M:3/4		     % Mesure
L:1/8		     % Duree de base pour une note
Q:1/4=90 	     % Tempo
K:F		     % Tonalite
\end{lstlisting}\vskip 0.3in
Puis le corps du format avec les informations sur les notes :\\
\begin{lstlisting}[language=Tex]
V:1	  	 % Numero de voix (1 = 1ere voix)
c f2 a/ g/ g/ f/ e/ f/ | c g2 b/ a/ a/ g/ ^f/ g/ |
V:2		 % Numero de voix (2 = 2eme voix)
F, C F, C F, C | E, C E, C E, C |
\end{lstlisting}
\vskip 0.6in
\newpage
Voici quelques significations des symboles présentés ci-dessus et qui peuvent être retrouvées dans la documentation :

\def\arraystretch{1.5}

\begin{center}
\begin{tabular}{>{\columncolor{grey}} c <{}|ccc}
\hline


\rowcolor{grey}\textbf{Symbole} & \textbf{Signification}\\

\hline


abcdefg & Nom des notes \\ \hline
ABCDEFG & Nom des notes un octave plus bas \\ \hline
|	& Séparation des mesures \\ \hline
z ou x	& Silence \\ \hline
V\string:chiffre & Numéro de voix de la prochaine ligne (maximum 10) \\ \hline
/[nombre] & Divise la durée de base de la note \\ \hline
[nombre] & Multiplie la durée de base de la note \\ \hline
,	& Diminue d'un octave \\ \hline
'	& Augmente d'un octave \\ \hline
\textasciicircum	& Ajout d'un dièse sur la prochaine note \\ \hline
=	& Ajout d'un bécarre sur la prochaine note \\ \hline
\_	& Ajout d'un bémol sur la prochaine note \\ \hline

\end{tabular}
\end{center}
\vskip 0.6in

\section{Bugs connus}
Voici, dans un ordre non particulier, une liste de bugs ou de comportements non souhaités :\\
\begin{itemize}
 \item Problème d'importation des fichiers ABC écrits sous Windows/Mac avec le retour à la ligne \textbf{\textbackslash r\textbackslash n} ou \textbf{\textbackslash r}. \\
 \item Quelques problèmes d'affichages avec les altérations notamment\\
 \item Crash lors de la modification d'une note qui est en train d'être jouée\\
\end{itemize}
Etant donné leurs structures, l'importation de fichiers .med est impossible d'une architecture à une autre (notamment 32/64 bits)

\chapter{Manuel d'utilisation}
\section{Interface générale}
L'interface se décompose en trois parties :\\
\begin{itemize}
 \item \textbf{Le menu} : Au clic de la souris pour ouvrir les sous-menus, on peut Créer, Ouvrir ou Sauvegarder une partition dans un fichier, ou encore effectuer des actions sur la partition. Certaines actions
 demandent à ce qu'un ou plusieurs éléments soient sélectionnés (Mesure, Note)  \\
 \item \textbf{La barre de toolbar} : Située juste en dessous du menu, elle permet de choisir le mode : A pour "Ajout", E pour "Edition"; de régler la durée des notes, les altérations, le volume, le tempo et est
 également composée d'un bouton Play/Pause pour jouer la partition.\\
 Lorsqu'une note est sélectionnée, on peut modifier ses caractéristiques directement en cliquant sur les élements de la toolbar :
 \end{itemize}
 \begin{center}
 \includegraphics[scale=0.45]{toolbar.png}\\
 \end{center}
 \newpage
 \begin{itemize}
 \item \textbf{Le corps} : Il permet de visualiser la partition. On peut la déplacer (Clic / Move), de zoomer/dézoomer, de sélectionner un élement, d'ajouter des notes, de supprimer des élements, etc.
 \end{itemize}
 \begin{center}
 \includegraphics[scale=0.30]{corps.png}\\
 \end{center}
 \vskip 0.2in


\subsection{Le mode Ajout}
Le mode \textbf{Ajout} est activé si la lettre \textbf{A} est affichée dans la barre de Toolbar, à gauche.\\
\begin{figure}[H]
  \centering
  \includegraphics[scale=0.80]{add.png}\\
    \caption{\textit{Symbole du mode ajout}}
\end{figure}
Ce mode permet l'ajout de notes sur la partition. Une fois ce mode activé, des zones bleutées apparaissent sur la partition au survol de la souris. 
Lors du clic dans une de ces zones, la note est rajoutée
au bon emplacement et à la hauteur du clic. On peut configurer les caractéristiques de la note ajoutée au préalable en sélectionnant sa durée, son altération dans la Toolbar. \\
Il est également possible de déplacer la partition et de zoomer.
\begin{figure}[H]
  \centering
  \includegraphics[scale=0.4]{ajout.png}\\
    \caption{\textit{La zone bleue délimite la zone d'ajout pour cette note}}
\end{figure}
\vskip 0.2in

\subsection{Le mode Edition}
Le mode \textbf{Edition} est activé si la lettre \textbf{E} est affichée dans la barre de Toolbar, à gauche.\\
\begin{figure}[H]
  \centering
  \includegraphics[scale=0.80]{edition.png}\\
    \caption{\textit{Symbole du mode édition}}
\end{figure}
Il permet de sélectionner un élément : une fois sélectionné, un rectangle légèrement transparent rouge indique de la bonne séléction d'une note, et un rectangle bleu indique 
de la sélection d'une mesure.\\ \\
Il est possible de sélectionner plusieurs éléments en même temps (Multi-Selection) en restant appuyé sur la touche CTRL. \\
Certaines actions ne peuvent être effectuées que si un seul élement est sélectionné
(par exemple l'ajout d'une mesure avant une autre). Pour déselectionner un élement, il suffit de cliquer dessus (toujours avec CTRL enfoncé).\\

\newpage
\begin{figure}[H]
  \centering
  \includegraphics[scale=0.45]{select.png}\\
    \caption{\textit{Des notes et une mesure sélectionnées}}
\end{figure}
\vskip 0.3in

Une fois un élément séléctionné on peut en fonction de son type : \\

Pour une mesure : \\
\begin{itemize}
 \item Supprimer la mesure avec Selection->Supprimer ou SUPPR : Supprime toutes les mesures en parallèle sur les autres portées.\\
 \item Régulariser la mesure (fusionner les silences quand cela est possible) avec Selection->Régulariser.\\
 \item Ajout des mesures avant ou après avec Ajouter->Mesures\\
 \item Changer l'armure avec Selection->Armures\\
 \item Changer la clé avec Selection->Clés\\
\end{itemize}
\vskip 0.2in
Pour une note/silence : \\
\begin{itemize}
 \item Supprimer la note avec Selection->Supprimer ou SUPPR (La transforme en silence de même durée)\\
 \item Diviser un silence ou une note en deux parties égales avec Selection->Diviser ou la touche d\\
 \item Changer la durée de la note avec la Toolbar\\
 \item Rajouter / Enlever une altération avec la Toolbar \\
\end{itemize}


\section{Explorateur de fichier}
Un explorateur de fichiers du disque est disponible. On peut notamment le retrouver lors de l'ouverture d'un fichier (Fichier->Ouvrir), ou lors de la sauvegarde (Fichier->Export en MED).\\\\
Seul les dossiers, les fichiers .med et les fichiers .abc sont affichés étant donné que seuls ces fichiers sont supportés par le programme.\\
Pour ouvrir un dossier, il suffit de double-clic dessus, pour sélectionner un fichier, un simple clic suffit et le colorise en bleu.\\ \\
Le bouton \textit{Parent} permet de monter d'un répertoire tandis que le bouton \textbf{Valider} valide le choix du fichier ou ne fait rien si aucun fichier n'est sélectionné.\\
Une fois un fichier validé, le programme détermine avec l'extension si c'est un fichier ABC ou MED et tente de l'ouvrir. Un message d'erreur s'affiche en cas d'échec.\\
\begin{figure}[H]
  \centering
  \includegraphics[scale=0.28]{explorer.png}\\
    \caption{\textit{Explorateur de fichier}}
\end{figure}
\vskip 0.2in

\section{Configuration Audio}
On peut y accéder en cliquant sur \textbf{Fichier/Configuration Audio}. Une fenêtre souvre permettant de choisir le type d'onde qui sera jouée (sinusoidale, carré, ou fusion des deux).\\
En dessous, on sélectionne les fréquences de notes souhaitées


\chapter{Perspectives} % Imporation/Export fichier MIDI, Compression format .med (Huffman...), 
Je compte bien évidemment continuer le développement du programme, qui n'est pour le moment qu'un prototype. Parmi les prochaines améliorations intéressantes qui pourraient avoir lieu :\\
\begin{itemize}
 \item Importation / Exportation des fichiers MIDI\footnote{Format MIDI : \url{http://www.jchr.be/linux/format-midi.htm}} : Ces fonctionnalités, proposées en tant qu'améliorations à la base n'ont pas été intégrées notamment par manque de temps. En effet, la spécification MIDI
 est complexe et il aurait été trop compliqué de mettre en place un système fiable dans les temps.\\
 \item Amélioration du format MED avec une compression : Par exemple, un codage de Huffman et une compression RLE. En effet, ce format contient (généralement) beaucoup de motifs commun et la compression
 pourrait être performante.\\
 \item Utilisation du format Soundfont et de FluidSynth\footnote{Documentation de FluidSynth : \url{https://sourceforge.net/p/fluidsynth/wiki/Home/}} (libre et gratuit) pour jouer des sons. Ce système ne permet pas de générer un son de fréquence libre mais permet
 l'utilisation simple de sons acoustiques de bonne qualité.\\
 \item Enfin, l'ajout des diverses fonctionnalités manquantes, par rapport à Musescore (symboles, transpositions, etc).\\
\end{itemize}

\end{document}
